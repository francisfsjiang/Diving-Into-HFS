\begin{XeClass}{BlockLocation}
\InsertFigure{gen/BlockLocation.pdf}{BlockLocation核心成员类图}
   
 \emph{BlockLocation}是块位置类.
 域包含主机号、端口号、拓扑路径以及偏移量和长度信息
 主要方法为域的get、set方法和实现Writable接口的读写方法

  \begin{XeMethod}{\XePublic\\ }{String[]}{getHosts}
       
 获取主机名列表

  \end{XeMethod}

  \begin{XeMethod}{\XePublic\\ }{String[]}{getNames}
       
 获取端口号列表

  \end{XeMethod}

  \begin{XeMethod}{\XePublic\\ }{String[]}{getTopologyPaths}
       
 获取每一个主机的网络拓扑路径的列表,路径的最后部分是主机

  \end{XeMethod}

  \begin{XeMethod}{\XePublic\\ }{long}{getOffset}
       
 获取块的偏移量

  \end{XeMethod}

  \begin{XeMethod}{\XePublic\\ }{long}{getLength}
       
 获取块的长度信息

  \end{XeMethod}

  \begin{XeMethod}{\XePublic\\ }{void}{setOffset}
       
 设置快的偏移量

  \end{XeMethod}

  \begin{XeMethod}{\XePublic\\ }{void}{setLength}
       
 设置块的长度信息

  \end{XeMethod}

  \begin{XeMethod}{\XePublic\\ }{void}{setHosts}
       
 设置当前块的主机名

  \end{XeMethod}

  \begin{XeMethod}{\XePublic\\ }{void}{setNames}
       
 设置当前块的端口号

  \end{XeMethod}

  \begin{XeMethod}{\XePublic\\ }{void}{setTopologyPaths}
       
 设置主机的网络拓扑路径

  \end{XeMethod}

  \begin{XeMethod}{\XePublic\\ }{void}{write}
       
 实现Writable的write方法,主要是将块位置的
 各参数信息写到输出缓存中

  \end{XeMethod}

  \begin{XeMethod}{\XePublic\\ }{void}{readFields}
       
 实现Writable的readFields方法
 读入块位置的各参数信息

  \end{XeMethod}

\end{XeClass}
