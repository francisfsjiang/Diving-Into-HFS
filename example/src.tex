\chapter{说明}

首先, 这个导出工具叫XGeb.
它依赖Eclipse JDT, 用于将Java源代码解析成逻辑结构.

\section{导出的先决条件}

注意: 当且仅当一份JavaDoc注释
属于类或方法
包含中文或其他非\emph{ACSII}字符
不包含``@@''
时, 该JavaDoc才会被导出.

完成复核之后, 不需要被导出的Javadoc, 在任意地方添加``@@''即可略过.

\section{XGen的内部机制与应对策略}

\section{JavaDoc的捕获}

XGen捕获JavaDoc的核心是, JDT的解析结果中, JavaDoc不会被纳入行号计数.
这意味着, 当一段JavaDoc和紧邻其后的类或方法之间没有空行的时候,
这个类或者方法的开始行数和JavaDoc的开始行数是相同的.

XGen两次遍历语法树, 每个pass处理的任务分别为
\begin{centering}
    \begin{enumerate}
        \item 捕获所有JavaDoc及其行号, 同时对JavaDoc进行预处理
        \item 捕获所有类/方法的定义, 尝试找出对应的JavaDoc
    \end{enumerate}
\end{centering}

所以, JavaDoc和紧邻其后的类或方法之间不要留空行.

\section{预处理}

XGen的预处理流程如下

\begin{centering}
    \begin{enumerate}
        \item 如果文本不含中文, 直接丢弃
        \item 删除注释用的/**, *和*/
        \item 删除所有独立tags, 比如@param
        \item 替换<code></code>和{@link Clazz{\#}method}成LaTeX中的\emph{}
    \end{enumerate}
\end{centering}

JavaDoc的原文会被全部导出, 因此需要移除非必要的文本.
注意, 正则表达式并不能完成解析HTML这样的工作, 所以请手动移除\emph{code}之外的其他标签.

\section{导出}

XGen导出的内容包括

\begin{centering}
    \begin{enumerate}
        \item JavaDoc正文
        \item 方法的modifier, 即public/private/protected
        \item 方法返回类型, 构造方法的返回类型视作当前类
    \end{enumerate}
\end{centering}

注意, 不导出参数列, 因为J系语言名字太长, 加上参数列根本没地方印.
因为不导出参数列, 所以并不能区分重载方法.
因为不区分重载方法, 所以, 重载方法的所有文档集合到一处便是.

\chapter{渲染示例}
\begin{XeClass}{FileSystem}

FileSystem :fs中最基本的抽象类,包含了文件系统最基本的操作抽象,
也包含了分布式文件的操纵抽象
继承了Configured类,提供了访问配置文件的方法
实现了Closeable接口,实现了关闭流的方法
1.域:
文件系统的缓存
键
文件系统的子类与统计信息的映射
统计信息
缓存中文件路径的集合
2.内部类:
Cache类:用来缓存文件系统
ClientFinalizer类:JVM关闭是调用进行清理
Key类:保存了与文件系统uri的信息,与文件系统对应
Statistics:用来记录统计信息的类
3.方法:
文件系统的关闭:
closeAll()
close()
文件系统的读取数据:
FSDataInputStream open(Path f, int bufferSize)
文件系统的写入数据:
FSDataOutputStream create(Path f)
FSDataOutputStream append(Path f)
重命名操作:
boolean rename(Path src, Path dst)
文件删除操作:
boolean delete(Path f)
文件或路径测试:
boolean exists(Path f)
boolean isDirectory(Path f)
boolean isFile(Path f)
文件复制操作:
copyFromLocalFile(Path src, Path dst)实现将文件从本地复制到其它路径
copyToLocalFile(Path src, Path dst)负责将FS下的文件复制到本地
文件移动操作:
moveFromLocalFile(Path[] srcs, Path dst)负责将本地文件移动到FS的其它位置
moveToLocalFile(Path src, Path dst)将FS下的文件移动到本地
文件查询:
FileStatus[] listStatus(Path f)
FileStatus[] globStatus(Path pathPattern)
通配格式:
interface PathFilter

    \begin{XeMethod}{\XePublic}{FileSystem}{get}

Get a filesystem instance based on the uri, the passed
configuration and the user
通过URI、配置对象和用户来获取缓存中的一个文件系统对象
具体过程为先判断用户字符串,若为空,则获取用户组信息的当前用户;
若不为空,则创建远程用户。
然后根据URI和配置对象返回文件系统对象

    \end{XeMethod}

    \begin{XeMethod}{\XePublic}{FileSystem}{get}

通过用户配置对象获取文件系统对象

    \end{XeMethod}

    \begin{XeMethod}{\XePublic}{URI}{getDefaultUri}

Get the default filesystem URI from a configuration.
通过配置对象获取默认URI

    \end{XeMethod}

    \begin{XeMethod}{\XePublic}{void}{setDefaultUri}

Set the default filesystem URI in a configuration.
通过配置对象和URI设置默认URI

    \end{XeMethod}

    \begin{XeMethod}{\XePublic}{void}{initialize}

Called after a new FileSystem instance is constructed.
通过URI和配置对象初始化,文件系统实例化后调用
for this FileSystem

    \end{XeMethod}

    \begin{XeMethod}{\XePrivate}{String}{fixName}

Update old-format filesystem names, for back-compatibility.  This should
eventually be replaced with a checkName() method that throws an exception
为了向下兼容,更新旧格式的文件系统名字

    \end{XeMethod}

    \begin{XeMethod}{\XePublic \\ \XeStatic}{LocalFileSystem}{getLocal}

Get the local file system
获取本地文件系统实例

    \end{XeMethod}


    \begin{XeInnerClass}{Cache}

Caching FileSystem objects
缓存文件系统对象的静态内部类
1.主要记录了Key和文件系统的映射,以及Key的集合
Key是Cache的内部类
主要记录了模式信息,授权信息以及用户组信息
2.其主要方法包括获取、删除、关闭Key所对应的文件系统实例

        \begin{XeInnerClass}{ClientFinalizer}

ClientFinalizer类为继承了Thread类
当Java虚拟机停止运行时,该线程才会启动
调用run进关闭清理工作

        \end{XeInnerClass}
        \begin{XeInnerClass}{Key}

FileSystem.Cache.Key

        \end{XeInnerClass}
    \end{XeInnerClass}
    \begin{XeInnerClass}{Statistics}

统计信息——静态内部类
主要包含了URI的模式信息
URI格式是scheme://authority/path
对HDFS文件系统,scheme是hdfs;对本地文件系统,scheme是file
该类包含了对读写的字节数、读取的操作次数内容进行的记录
被用于\emph{AbstractFileSystem}和\emph{FileSystem.}中的统计信息。
Statistics主要记录读写字节数。

    \end{XeInnerClass}
\end{XeClass}

