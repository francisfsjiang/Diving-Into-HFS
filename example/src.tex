\chapter{说明}

首先, 这个导出工具叫XGeb.
它依赖Eclipse JDT, 用于将Java源代码解析成逻辑结构.

\section{导出的先决条件}

注意: 当且仅当一份JavaDoc注释
属于类或方法
包含中文或其他非\emph{ACSII}字符
不包含``@@''
时, 该JavaDoc才会被导出.

完成复核之后, 不需要被导出的Javadoc, 在任意地方添加``@@''即可略过.

\section{XGen的内部机制与应对策略}

\section{JavaDoc的捕获}

XGen捕获JavaDoc的核心是, JDT的解析结果中, JavaDoc不会被纳入行号计数.
这意味着, 当一段JavaDoc和紧邻其后的类或方法之间没有空行的时候,
这个类或者方法的开始行数和JavaDoc的开始行数是相同的.

XGen两次遍历语法树, 每个pass处理的任务分别为
\begin{centering}
    \begin{enumerate}
        \item 捕获所有JavaDoc及其行号, 同时对JavaDoc进行预处理
        \item 捕获所有类/方法的定义, 尝试找出对应的JavaDoc
    \end{enumerate}
\end{centering}

所以, JavaDoc和紧邻其后的类或方法之间不要留空行.

\section{预处理}

XGen的预处理流程如下

\begin{centering}
    \begin{enumerate}
        \item 如果文本不含中文, 直接丢弃
        \item 删除注释用的/**, *和*/
        \item 删除所有独立tags, 比如@param
        \item 替换<code></code>和{@link Clazz{\#}method}成LaTeX中的\emph{}
    \end{enumerate}
\end{centering}

JavaDoc的原文会被全部导出, 因此需要移除非必要的文本.
注意, 正则表达式并不能完成解析HTML这样的工作, 所以请手动移除\emph{code}之外的其他标签.

\section{导出}

XGen导出的内容包括

\begin{centering}
    \begin{enumerate}
        \item JavaDoc正文
        \item 方法的modifier, 即public/private/protected
        \item 方法返回类型, 构造方法的返回类型视作当前类
    \end{enumerate}
\end{centering}

注意, 不导出参数列, 因为J系语言名字太长, 加上参数列根本没地方印.
因为不导出参数列, 所以并不能区分重载方法.
因为不区分重载方法, 所以, 重载方法的所有文档集合到一处便是.

\chapter{渲染示例}
\begin{XeClass}{AbstractFileSystem}
AbstractFileSystem(下简称AFS)是提供给Hadoop FS体系下的具体文件系统实现者的接口.AFS以抽象类的形式向Hadoop FS中的具体文件系统的实现者提供接口.Hadoop FS的架构与Unix下VFS类似, 应用不直接访问AFS实例,而是通过\emph{FileContext}实例来访问文件系统中的文件.传递给AFS的路径(pathnames)可以是符合具体文件系统规定的、完全限定的URI,也可以是以给定文件系统的根目录为'/'目录的, 以‘/’分隔的路径(同Unix体系).由于AbstractFileSystem本身是提供给文件系统实现者的接口,且Hadoop FS中对文件的访问全部由\emph{FileContext}控制, 所以,这个类里没有public方法, 而是提供了三类protected方法:AFS实例由工厂方法\emph{AbstractFileSystem.get(URI, Configuration)}创建.AFS中封装了一批通用操作的抽象方法, 这些方法由具体文件系统的实现者实现.

    \begin{XeMethod}{\XeStatic}{T}{newInstance}
        反射机制实现的初始化, URI和Configuration会被传递给实际的构造方法
    \end{XeMethod}

    \begin{XeMethod}{\XeStatic \\ \XePrivate}{AbstractFileSystem}{createFileSystem}
        通URI和Configuration创建AFS实例,该方法被直接用于工厂方法\emph{AbstractFileSystem.get(URI, Configuration)}
    \end{XeMethod}

    \begin{XeMethod}{\XeStatic}{AbstractFileSystem}{get}
        创建AbstractFileSystem实例的主要工厂方法, 通过URI的体系(scheme)和权限获取文件系统.URI中描述的体系确定配置中一个名为\emph{fs.AbstractFileSystem.scheme.impl}的属性, 这个属性的值就是该scheme对应的AbstractFileSystem类的具体实现.完整的URI和配置信息会被传递给AbstractFileSystem的工厂方法, 即\emph{AbstractFileSystem.createFileSystem(URI, Configuration)}
    \end{XeMethod}

    \begin{XeMethod}{\XeProtected}{AbstractFileSystem}{AbstractFileSystem}
        子类需用的构造方法, 该方法配置实例的URI和统计信息.
    \end{XeMethod}

    \begin{XeMethod}{\XePrivate}{URI}{getUri}
        由给定的URI中截取文件系统根目录的URI.
    \end{XeMethod}

\end{XeClass}

