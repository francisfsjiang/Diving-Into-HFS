% !TEX root = ../report.tex
%
\chapter{\HadoopFS 移植}
\label{ch:trans}

%%%%%%%%%%%%%%%%%%%%%
%
% section: 尝试
%
\section{尝试}
\label{sec:trans:try}
对{\HFS}的操作通过``hadoop fs $<$arguments$>$...''指令执行。
hadoop脚本实际调用启动Java虚拟机运行org.apache.hadoop.fs.FsShell类。
{\HFS}上的文件操作通过这个类封装给用户。我们在{\Windows}环境下直接运行这个类来进行移植尝试。
根据Usage,我们依次尝试命令,对出现的问题进行修改。

%%%%%%%%%%%%%%%%%%%%%
%
% subsection: -ls
%
\subsection{-ls}
\label{subsec:trans:ls}
如图\ref{fig:trans:ls0},``-ls''的运行抛出异常。

\InsertFigure[fig:trans:ls0]{trans/run-ls0}{``-ls''测试结果}

检查调用栈发现问题在于org.apache.hadoop.fs.RawLocalFileSystem\$RawLocalFileStatus类的loadPermissionInfo方法中想要调用``ls -ld''命令,获取当前目录的权限、文件拥有者、文件所在的组。
相关代码如下:

\InsertListing[caption=org/apache/hadoop/fs/RawLocalFileSystem.java(loadPermissionInfo)]{RawLocalFileSystem0.java}{465}{482}

\InsertListing[caption=org/apache/hadoop/util/Shell.java]{Shell0.java}{68}{72}

原生{\Windows}环境并没有这样的命令,并且文件管理系统也和{\Unix}有出入。
对照一些{\Windows}下的{\Unix}工具移植,对方法进行修改后,执行成功。
相关代码和运行结果如下:

\InsertListing[caption=org/apache/hadoop/fs/RawLocalFileSystem.java(loadPermissionInfo)]{RawLocalFileSystem1.java}{467}{497}

\InsertListing[caption=org/apache/hadoop/util/Shell.java]{Shell1.java}{68}{69}

\InsertFigure{trans/run-ls1}{修改后``-ls''测试结果}

%%%%%%%%%%%%%%%%%%%%%
%
% subsection: -cp
%
\subsection{-cp}
\label{subsec:trans:cp}

如图\ref{fig:trans:cp0},``-cp''异常结束,只有一条输出到stderr中的错误信息。

\InsertFigure[fig:trans:cp0]{trans/run-cp0}{``-cp''测试结果}

在DEBUG环境下运行命令,最终追踪到org.apache.hadoop.fs.RawLocalFileSystem下的一处调用,相关代码如下:

\InsertListing[caption=org/apache/hadoop/fs/RawLocalFileSystem.java(setPermission)]{RawLocalFileSystem1.java}{536}{544}

\InsertListing[caption=org/apache/hadoop/util/Shell.java]{Shell1.java}{59}{60}

RawLocalFileSystem类的setPermission方法中想要调用``chmod 00777''命令设置目标文件权限。
经过尝试,我们修改源码检测运行操作系统,在{\Windows}操作系统下不执行chmod命令。
修改后的代码和运行结果如下:

\InsertListing[caption=org/apache/hadoop/fs/RawLocalFileSystem.java(setPermission)]{RawLocalFileSystem2.java}{536}{546}

\InsertFigure{trans/run-cp1}{修改后``-cp''测试结果}

%%%%%%%%%%%%%%%%%%%%%
%
% subsection: -getmerge
%
\subsection{-getmerge}
\label{subsec:trans:getmerge}
对``-getmerge''的测试虽然成功结束,但没有输出文件。
经过分析,我们发现了一个Bug。

\InsertListing[caption=org/apache/hadoop/fs/FileUtil.java(copyMerge)]{FileUtil0.java}{259}{267}

266行的if语句条件应取反,修正如下

\InsertListing[caption=org/apache/hadoop/fs/FileUtil.java(copyMerge)]{FileUtil1.java}{259}{267}

%%%%%%%%%%%%%%%%%%%%%
%
% section: 结论
%
\section{结论}
\label{sec:trans:conclusion}
我们的尝试表明,让{\HFS}支持原生{\Windows}下的文件系统是可行的。
核心问题在于{\HFS}的实现包含大量对类{\Unix}操作系统中二进制实用工具的使用,而没有考虑{\Windows}系统上的使用。