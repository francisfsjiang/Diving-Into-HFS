% !TEX root = ../report.tex
%
\chapter{综述}
\label{ch:intro}

\section{题目}

\Hadoop 项目文件系统源代码分析

\section{问题回顾}
\label{sec:intro:review}

根据实验手册, 本题目共有如下六项需求:
\begin{XeEnum}
    \item 阅读前期分析报告, 并仔细阅读与修正.
    \item 构建阅读代码的环境.
    \item 在详细阅读源代码的基础上, 理解设计思想, 并给出软件架构图.
    \item 在软件架构图的基础上, 给出源代码的详细类图和核心过程的顺序图,
    说明源代码的设计如何与设计思想呼应.
    \item 在明确详细设计的基础上, 以源代码中的相关代码为例证明自己的判断.
    \item 通过阅读源代码, 深刻体会\Hadoop 云计算平台与文件系统之间的关系,
          修正提供的分析报告, 给出完整的分析过程与修订后的报告.
\end{XeEnum}

\section{我们的工作}
\label{sec:intro:result}

\begin{XeEnum}
    \item 尝试构建运行\Hadoop 0.21版本
    \item 配置分析环境, 并重点分析了\Hadoop 0.21中的\HadoopFS 部分的源代码
    \item 在代码分析的基础上给出架构说明文档
    \item 在2, 3的基础上将\HadoopFS 移植到Windows平台
    \item 对原始报告进行了修订, 并重新编写本册实验报告
\end{XeEnum}

\section{报告结构}
\label{sec:intro:structure}

\begin{XeStructure}
    \space \\
    \XeStructureItem{ch:intro}{报告综述}
    \XeStructureItem{ch:hfs-intro}{\HadoopFS 的介绍}
    \XeStructureItem{ch:hfs}{\HadoopFS 架构分析}
    \XeStructureItem{ch:src}{详尽的源代码分析}
    \XeStructureItem{ch:env}{代码阅读环境和构建环境的搭建}
    \XeStructureItem{ch:trans}{\HadoopFS 0.21版本向Windows移植的经历}
\end{XeStructure}
