\chapter{\HadoopFS 综述}
\label{ch:hfs-intro}

\cleanchapterquote{The Apache Hadoop project develops open-source software
for reliable, scalable, distributed computing.}{hadoop.apache.org}{引自项目首页}

\section{什么是\Hadoop}
\label{ch:hfs-intro:hadoop}

\Hadoop 是目前最流行的分布式计算平台之一.
所谓``分布式计算平台'', 其中的关键词有二:
其一是``分布式计算''. ``分布式''意味着\Hadoop 通常将运行在规模不定的网络上,
通过网络中各个节点的协作来处理大规模的计算; 而要在这样的环境下实现``计算'',
首先要解决的几个问题是
\begin{XeEnum}
    \item\label{it:hi:p1} 如何将计算任分解成子问题并务分散到规模不定的集群上
    \item\label{it:hi:p2} 如何合并来自不同计算节点之间的计算结果并求解最终结
    \item\label{it:hi:p3} 如何在分布式环境为上层应用提供相应的文件服务, 用以支持\Hadoop 系统的运行.
\end{XeEnum}
其中, 为了解决问题~\ref{it:hi:p1}和问题~\ref{it:hi:p2},
\Hadoop 使用了\MapReduce 的策略;
为了解决问题~\ref{it:hi:p3}, \Hadoop 实现了强大的\Hadoop 分布式文件系统(\HadoopDFS, \HDFS).
\MapReduce 和\HDFS 是\Hadoop 平台的核心设计.

其二是``平台''. 平台意味着\Hadoop 系统本身提供的是通用机制, 而不局限于具体策略.

\section{什么是\HadoopFS}
\label{ch:hfs-intro:hfs}

\Hadoop 文件系统(\HadoopFS, \HFS)是为\Hadoop 云计算平台提供文件服务的一个抽象层.
按照Wikipedia的解释, 文件系统是
\begin{quote}
The structure and logic rules used to manage the groups of information and their names is called a "file system".
\end{quote}
即一组用于
\begin{XeEnum}
    \item 管理信息
    \item 为信息提供命名索引
\end{XeEnum}
的结构和逻辑规则.

对\Hadoop 而言, \HFS 的身份是服务提供者, 和一个基础抽象层.

\subsection{作为服务提供者的\HFS}

服务提供者是\HFS 在\Hadoop 中的角色,
其主要职责是向上层的\Hadoop 应用提供可靠的文件访问服务.

\subsection{作为抽象层的\HFS}

根据\HadoopFS 中的\emph{AbstractFileSystem}类的文档
\begin{quote}
This class provides an interface for implementors of a \HadoopFS
(analogous to the \VFS\space of \Unix). Applications do not access this class;
instead they access files across all file systems using \emph{FileContext}.
\end{quote}
指出的``\emph{AbstractFileSystem}以抽象类的形式向\HFS 中的具体文件系统的实现者提供接口,
与\Unix 下\VFS 的机制类似''.

而\VFS 本身是
\begin{quote}
    \ldots an abstraction layer on top of a more concrete file system
\end{quote}

类似地, 我们也可以这样描述\HFS
\begin{quote}
    \HFS 是建立在不同的具体文件系统智商的抽象层.
\end{quote}

\blankfigure{}

如~\ref{ch:hfs-intro:hfs}节所述, \HFS 是用于向\Hadoop 提供文件服务的,
横断于各个具体文件系统之上的抽象层.
HAL屏蔽底层硬件之间的差异, 向上提驱动磁盘的接口;
具体文件系统屏蔽不同体系(如NTFS, ext, xfs, btfs, etc.)下信息在物理磁盘上的组织方式的差异,
向上提供操纵文件的接口;
而抽象文件系统, 负责屏蔽不同的具体文件系统之间的差异(如命名管理, 权限控制等),
向上提供完全统一的文件操纵接口. 这个最高的抽象层, 就是HFS所处的位置.

\section{\HFS 与\HDFS}

\HDFS 的知名度远要比\HFS 高的多.

\HadoopFS 是~\ref{ch:hfs-intro:hfs}节中描述的一个宏观的抽象文件系统,
而\HadoopDFS 只是该抽象文件系统管理下的一种具体文件系统的实现.

在\HFS 中, 处于继承树顶端且有着相当高的抽象程度的基类\FS 的文档中记录到:

\begin{quote}
(\FS\space is) an abstract base class for a fairly generic filesystem. It
may be implemented as a distributed filesystem, or as a "local"
one that reflects the locally-connected disk.
\end{quote}

其中所指的分布式文件系统的实例之一, 就是\HDFS.

