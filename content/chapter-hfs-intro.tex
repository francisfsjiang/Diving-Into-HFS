\chapter{\HadoopFS 综述}
\label{ch:hfs-intro}

\cleanchapterquote{The Apache Hadoop project develops open-source software
for reliable, scalable, distributed computing.}{hadoop.apache.org}{引自项目首页}

\section{什么是\Hadoop}
\label{ch:hfs-intro:hadoop}

\Hadoop 是一个时下流行的云计算平台. 

\section{什么是\HadoopFS}
\label{ch:hfs-intro:hfs}

\Hadoop 文件系统(\HadoopFS, \HFS)是为\Hadoop 云计算平台提供文件服务的一个抽象层.
按照Wikipedia的解释, 文件系统是
\begin{quote}
The structure and logic rules used to manage the groups of information and their names is called a "file system".
\end{quote}
即一组用于
\begin{XeEnum}
    \item 管理信息
    \item 为信息提供明命名索引
\end{XeEnum}
的结构和逻辑规则.

对\Hadoop 而言, \HFS 的身份是服务提供者, 和一个基础抽象层.

\subsection{作为服务提供者的\HFS}

服务提供者是\HFS 在\Hadoop 中的角色,
其主要职责是向上层的\Hadoop 应用提供可靠的文件访问服务.

\subsection{作为抽象层的\HFS}

根据\HadoopFS 中的\emph{AbstractFileSystem}类的文档
\begin{quote}
This class provides an interface for implementors of a \HadoopFS
(analogous to the \VFS\space of \Unix). Applications do not access this class;
instead they access files across all file systems using \emph{FileContext}.
\end{quote}
指出的``\emph{AbstractFileSystem}以抽象类的形式向\HFS 中的具体文件系统的实现者提供接口,
与\Unix 下\VFS 的类似''.

明确指出了\HFS 是一个与\Unix 下的\VFS 十分类似的系统,
而\VFS 本身是
\begin{quote}
    \ldots an abstraction layer on top of a more concrete file system
\end{quote}

类似地, \HFS 是用于向\Hadoop 提供文件服务的, 横断于各个具体文件系统之上的抽象层. 它
HAL屏蔽底层硬件之间的差异,
向上提驱动磁盘的接口,
具体文件系统屏蔽不同体系(如NTFS, ext, xfs, btfs, etc.)下信息在物理磁盘上的组织方式的差异,
向上提供操纵文件的接口,
而抽象文件系统,
负责屏蔽不同的具体文件系统之间的差异(如命名管理, 权限控制等),
向上提供完全统一的文件操纵接口. 这个最高的抽象层, 就是HFS和VFS所处的位置.

\section{\HFS 与\HDFS}

\HadoopFS 是~\ref{ch:hfs-intro:hfs}节中描述的一个宏观的抽象文件系统,
而\HadoopDFS 只是该抽象文件系统管理下的一种具体文件系统的实现.

在\HFS 中, 处于继承树顶端且有着相当高的抽象程度的基类\FS 的文档中记录到:
\begin{quote}
(\FS is) an abstract base class for a fairly generic filesystem. It
may be implemented as a distributed filesystem, or as a "local"
one that reflects the locally-connected disk.
\end{quote}
其中所指的分布式文件系统的实例之一, 就是\HDFS.

