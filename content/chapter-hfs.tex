\chapter{\HFS 总体架构}
\label{ch:hfs}

\section{\HFS 架构}
\label{sec:hfs:arch}

\subsection{The big picture}
\label{ssec:hfs:bigpic}

如\XeSecRef{ssec:hfs:al}中所述, {\HFS}与{\VFS}的结构十分相似,
因此, 如果将下起硬件上至{\HFS}的视作一个完整的系统, 则{\HFS}构成一个典型的层次结构,
状如下图:

\blankfigure{参考VHF的架构图}

这是一个在层层抽象的基础上构建起来系统.

% TODO 如果将考察的范围缩小到兼容层本身

\subsection{作为抽象层}
\label{ssec:hfs:al}

从硬件抽象层的角度理解{\HFS},
{\HFS}在实现级别上使用了类似于Command模式和Adapter模式的策略,
即通通过如下继承的层级结构

\blankfigure{抽象继承结构}

来实现将统一的, 高度抽象的文件操作映射到各个具体文件系统之上的目的.

在{\HFS}参与管理的诸多具体文件系统之中, 最特殊的一个具体文件系统的实现莫过于{\HDFS}.
\XeSecRef{sec:hfs:fs}中粗略讨论了{\HDFS}的宏观特性,
但是, 无论{\HDFS}所拥有的分布式特性与传统的文件系统, 或者, 按照官方文档中的用法, 本地文件系统,
有着多么大的差异, {\DHFS}中的格式操作依然可以被映射到{\HFS}体系中的基本抽象上.

% TODO 验证文件系统映射的具体实现

\subsection{作为服务提供者}
\label{ssec:hfs:srv}

从服务提供者的角度理解{\HFS}, 则{\HFS}本身通过URI的形式向其上层应用提供基础文件访问操作,
以及流式访问操作.

由于{\Hadoop}系统中, {\HFS}体系所需要支持的重中之重是{\HDFS}, 因此,
在{\HFS}的最高层抽象结构之中, 就已经有了对与冗余数据节点的支持,
具体细节将在\XeSecRef{sec:hfs:fs}和\XeSecRef{sec:hfs:afs}中讨论.

\section{核心模块}
\label{sec:hfs:modules}

如果以包为单位来考察{\HFS}, 且仅仅考虑与{\HFS}自身相关的代码,
那么这张粗粒度的表格中几乎没有什么表项:
\begin{XeDuoLineTabular}{包}{作用}
    \XeDLTItem{fs}{文件系统相关的大部分代码}
    \XeDLTItem{io}{I/O相关的代码}
    % TODO 脑补一下这张表里应该有啥
\end{XeDuoLineTabular}

如果以更细粒度的逻辑结构来划分``模块''之间的关系, 那么{\HFS}的逻辑功能块的划分应该是这个样子:
\begin{XeDuoLineTabular}{模块}{功能}
    \XeDLTItem{AbstractFileSystem族系}{各个具体文件系统的接口和实现}
    \XeDLTItem{FileSystem族系}{文件系统用户及其接口定义}
\end{XeDuoLineTabular}

\section{{\FiS}族系}
\label{sec:hfs:fs}

{\FiS}(下简称{\FS})是一般的通用文件系统的抽象基类.
FS可以被实现为分布式的文件系统, 如{\HDFS}, 亦或是一个本地的文件系统,
所有具有访问{\HDFS}的潜在可能的用户代码都应该被封装在一个{\FiS}实例里.
{\HDFS}是一个对外表现如同单一磁盘的多机系统,
它的容错能力和对大容量存储的支持使得{\HDFS}尤为实用.
{\FiS}的本地实现是{\emph LocalFileSystem},
分布式实现是{\emph DistributedFileSystem}(在{\HDFS}一支).

{\FiS}是{\HadoopFS}早期版本中唯一的抽象基类, 但在0.21版本中,
因为{\AbsFS}的引入, 使得{\FiS}的意义发生了一定变更,
详见\XeSecRef{sec:hfs:afs}.

\section{\AbsFS 族系}
\label{sec:hfs:afs}

{\AbsFS}是提供给{\HadoopFS}体系下的具体文件系统实现者的接口.
{\AbsFS}以抽象类的形式向{\HadoopFS}中的具体文件系统的实现者提供接口.
{\HadoopFS}的架构与{\Unix}下{\VFS}类似,
应用不直接访问{\AbsFS}实例, 而是通过{\emph FileContext}实例来访问文件系统中的文件.
传递给{\AbsFS}的路径(pathnames)可以是符合具体文件系统规定的, 完全限定的URI,
也可以是以给定文件系统的根目录为'/'目录的, 以‘/’分隔的路径(同{\Unix}体系).

由于{\AbsFS}本身是提供给文件系统实现者的接口,
且{\HadoopFS}中对文件的访问全部由{\emph FileContext}控制,
所以, 这个类里没有{\emph public}方法, 而是提供了5类{\emph protected}方法:

\begin{XeItem}
    \item 构造方法(针对子类)和工厂方法(针对用户)
    \item 统计信息, {\emph FsStatus}记录容量和空间使用情况,
          {\emph Statistics}记录读写字节计数.
    \item 文件操作, 如读, 写, 增, 删, 重命名, 权限设置等抽象方法, 交由文件系统开发者实现.
    \item URI验证
    \item 校验和
\end{XeItem}

这些方法描述了{\HadoopFS}的最高层抽象.

AFS实例由工厂方法{\emph AbstractFileSystem.get}创建.
{\AFS}中封装了一批通用操作的抽象方法, 这些方法由具体文件系统的实现者实现.

{\emph \Hadoop 源代码分析报告}指出:

\begin{quote}
    {\FiS}类是0.21版本之前唯一的基类, 但在0.21版本中, 出现了{\AbsFS},
    该类似乎来取代{\FiS}类原来的部分功能. 在这两个基类的基础上形成了两个类继承的层次结构。
\end{quote}

现在, 在更详尽分析的基础上,
我们可以对{\FiS}和{\AFS}的关系给出一个更精准的描述,
即, 引入{\AFS}的目的是为了分解更清晰的C/S架构.
% TODO 更改这C/S

两个类各自的文档中有如下表述:

\begin{quote}
    {\AbsFS}是提供给具体文件系统实现者的接口.
\end{quote}

\begin{quote}
    {\FiS}是一般的通用文件系统的抽象基类.
    所有具有访问{\HDFS}的潜在可能的用户代码都应该被封装在一个{\FiS}实例里。
\end{quote}

在实现层面上, 对比{\AFS}和{\FS}, 可以发现:

\begin{XeItem}
    \item AFS有比FS更强大的URI解析支持
    \item AFS独有对Scheme的处理
    \item AFS中提供的具体文件操作比FS更少, 几乎仅仅提供了最通用的抽象接口
    \item 因此, 引入{\AbsFS}的目的不是部分取代, 而是全面分流.
\end{XeItem}

历史上, AFS的出现晚于{\FiS}, 因此, 在{\emph release 0.21}这个版本中,
采取了两种兼容措施:

其一是, 在继承树中, 为了避免与{\FiS}冲突, {\AbsFS}的直系后裔以``Fs''结尾;
其二是, 为{\AbsFS}添加了针对{\FiS}的适配器{\emph DelegateToFileSystem},
用以复用原有代码. 在这个版本中, {\emph DelegateToFileSystem}被大量使用,
以期耗费尽量低的成本而快速完早期实现.

此外, 通过对目录的扫描, 可以发现, 几乎所有{\AbsFS}族系的类都没有真正的进入具体实现的体系,
按照前文中的功能对比, 大量应该被剥离出去的代码也没有真正的分离出去.

\section{FileContext族系}
\label{sec:hfs:fc}
